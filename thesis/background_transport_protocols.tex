\section{Transport protocols}
\subsection{TCP}
\subsection{UDP}
\subsubsection{Message- oriented protocols}
\subsection{RTP and RTCP}
\subsection{SCTP}
SCTP is a transport protocol offering many of the same services as TCP while also bringing additional features and improvements to the table. 
SCTP offers a point- to- point connection- oriented reliable delivery service while also using the same flow and congestion control algorithms as TCP.   %cite RFC perhaps
As opposed to TCP, SCTP is message- oriented. %explain more
An SCTP connection is called an association.

SCTP separates application data into chunks, each identified by a separate chunk header. 
These chunks are bundled into a single SCTP message that consists of an SCTP message header followed by several data chunks.
A key feature here is that the data chunks are independently identified with a separate header, thus a single SCTP message can contain data from separate streams of application data. For example one stream being text messages and another being the transfer of a file in a messaging application.
The advantage of this packet structure is that it means SCTP can support multi- streaming since it can send multiple data streams in parallel through a single SCTP association. 
%TODO: have a small section explaining head- of- line blocking and how SCTP avoids it 

Multi- streaming means that an application can transmit several independent streams of data in parallel. 

SCTP separates application data into chunks, each identified by a separate header. 
A single SCTP packet can contain several data chunks from different application data streams.

%TODO:cite some of the papers that have shown SCTP improvements compared to TCP
