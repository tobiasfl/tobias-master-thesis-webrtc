\documentclass[UKenglish]{ifimaster}  %% ... or USenglish or norsk or nynorsk
\usepackage[utf8]{inputenc}           %% ... or latin1
\usepackage[T1]{fontenc,url}
\urlstyle{sf}
\usepackage{babel,textcomp,csquotes,duomasterforside,varioref,graphicx}
\usepackage[backend=biber,style=numeric-comp]{biblatex}

\title{My title here}        %% ... or whatever
\subtitle{Managing Real- Time Video and Data Flows with Coupled Congestion Control Mechanism}         %% ... if any
\author{Tobias Fladby}                      %% ... or whoever 

\addbibresource{./bibdb/mybib.bib}            %% ... or whatever

\begin{document}
\duoforside[dept={Department of Informatics},   %% ... or your department
  program={programming and system architecture},  %% ... or your programme
  long]                                        %% ... or long

\frontmatter{}
\chapter*{Abstract}                   %% ... or Sammendrag or Samandrag

\tableofcontents{}
\listoffigures{}
\listoftables{}

\chapter*{Preface}                    %% ... or Forord

\mainmatter{}
\part{Introduction}                   %% ... or Innledning or Innleiing

\chapter{Background}
\section{Real- time communication protocol requirements}
Video data by nature is large in size so transmitting it creates a lot of traffic. 
This makes real- time communication challenging because it requires low latency in order to assure a good user experience. 
History and previous research [cite relevant stuff, like congestion collapse]has shown that protocols should employ mechanisms that limit the amount of data sent per second to a reasonable level in order to avoid congestion as well as keep the latency low.

\section{WebRTC}
\section{Google Congestion Control}
RTP by itself only provides simple end- to- end delivery services for multimedia[cite RTP standard], since real- time communication requires congestion control it must implemented on top of RTP. 
Chromium's WebRTC implementation uses an algorithm called Google Congestion Control \cite{rfc8699} to provide the mechanism.
It consists of two controllers, one loss- based and one delay- based. 
The loss- based controller located on the sender- side, uses loss rate, RTT and REMB[Cite REMB message definition] messages to compute a target sending bitrate. 
The delay- based controller can either be implemented on the receiver- side or sender- side.
It uses packet arrival info to compute a maximum bitrate which is passed to the loss- based controller. The actual sending rate is set to the minimum of the two bitrates.
% Might want to explain RTP terms like groups of packets and etc. 
\subsection{The loss- based controller}
The loss- based controller is run every time a feedback message from the receiver- side is received. 
If more than 10\% of packets have been lost when feedback is received the controller decreases the estimate. 
If less than 2\% is lost it will increase the estimate under the presumption that there is more bandwidth to utilize. 
Otherwise the estimate stays the same.
\subsection{The delay- based controller}
 The delay- based controller consists of several parts: pre- filtering, an arrival- time filter, an over- use detector and a rate controller. %% probably smart to have a figure of this

 Pre- filtering is used to make sure that channel outages, events unrelated to congestion are not interpreted as congestion.
Packets will naturally be delayed when a channel outage occurs so without this filter the algorithm would unnecessarily lower bitrate, thus lowering the quality of the communication for no reason.
A channel outage will cause the packets to be queued in network buffers, thus when the channel is restored the packets will arrive in bursts. 
The filter utilizes the fact that the packet groups will arrive in bursts during a channel outage and merges them under such conditions.

The arrival- time filter is responsible for calculating the queueing time variation which is an estimation of how the delay is developing at a certain time.
%TODO: fill this one with more info
The goal of the over- use detector is to produce a signal that drives the state of the remote rate controller. 
The goal of the over- use detector is to compare the queueing time variation obtained as output from the arrival- time filter with a threshold. If the estimate is above the threshold for a certain amount of time and not sinking it wil signal the rate contol.
%TODO: incorporate explaination of the remote rate controller as well as the related state machine and how it works

\section{SCTP}
\part{The project}                    %% ... or ??

\chapter{Planning the project}        %% ... or ??


\part{Conclusion}                     %% ... or Konklusjon

\chapter{Results}                     %% ... or ??


\backmatter{}
\printbibliography
\end{document}
