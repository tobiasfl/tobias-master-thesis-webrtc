\documentclass[UKenglish]{ifimaster}
\usepackage[utf8]{inputenc}           %% ... or latin1
\usepackage[T1]{fontenc,url}
\urlstyle{sf}
\usepackage{babel,textcomp,csquotes,duomasterforside,varioref,graphicx}
\usepackage[backend=biber,style=numeric-comp]{biblatex}
\usepackage[hidelinks]{hyperref}

\usepackage{algorithm}
\usepackage{algorithmic}

\title{Awesome title here}       
\subtitle{Managing Real- Time Video and Data Flows with Coupled Congestion Control Mechanism}
\author{Tobias Fladby}                     

\addbibresource{./bibdb/mybib.bib}           
\addbibresource{./bibdb/allrefs.bib}

\begin{document}
\duoforside[dept={Department of Informatics},   %% ... or your department
  program={programming and system architecture},  %% ... or your programme
  long]                                       

\frontmatter{}
\chapter*{Abstract}            

\tableofcontents{}
\listoffigures{}
\listoftables{}

\chapter*{Preface}              

\mainmatter{}
\chapter{Introduction}              

\section{Problem statement}
\section{Contributions}
\section{Research questions}
\emph{Overall:}
\begin{itemize}
    \item Can two heterogenous control mechanisms be coupled? Will it improve overall performance?
\end{itemize}
\emph{Simplicity:}
\begin{itemize}
    \item Can such a mechanism be designed simple enough for widespread implementation? Moreover can it be easily integrated with other congestion control mechanisms? 
\end{itemize}
\emph{Fairness:}
\begin{itemize}
    \item Will both flows get their allocated share of bandwidth when needed?
    \item Will the coupled flows be fair to other flows sharing the same bottleneck?
    \item Will the bandwidth be shared according to configured priority?
\end{itemize}
\emph{Delay:}
\begin{itemize}
    \item Can it reduce delay spikes?
\end{itemize}
\emph{Link utilization:}
\begin{itemize}
    \item Will link utilization be equal to a single flow using the full link?
\end{itemize}
\emph{Responsiveness:}
\begin{itemize}
    \item Will any of the congestion control mechanisms be more responsive to congestion in the network?
\end{itemize}
\emph{Packet loss:}
\begin{itemize}
    \item Will any of the flows experience less packet loss?
    \item Can it reduce packet loss spikes for the flows?
\end{itemize}

\section{Organization}

\chapter{Background}

\section{WebRTC architecture}
\subsection{Real- time communication}
\subsection{Standardization}
\subsection{Protocol Stack}
\subsection{User API}
\subsubsection{Services}
\subsubsection{RTCPeerConnection API}
\subsubsection{DataChannel API}
\subsection{Signalling}
\subsubsection{NAT}
\subsubsection{ICE Framework}
\subsubsection{TURN}
\subsubsection{STUN}
\subsubsection{SDP}
\subsection{Encryption}
\subsubsection{TLS}
\subsubsection{DTLS}
\subsection{Usage}
\subsection{Browser engine}
\subsubsection{Aquiring WebRTC statistics}

\section{Transport protocols}
\subsection{TCP}
\subsection{UDP}
\subsubsection{Message- oriented protocols}
\subsection{RTP and RTCP}
\subsection{SCTP}
SCTP is a transport protocol offering many of the same services as TCP while also bringing additional features and improvements to the table. 
SCTP offers a point- to- point connection- oriented reliable delivery service while also using the same flow and congestion control algorithms as TCP.   %cite RFC perhaps
As opposed to TCP, SCTP is message- oriented. %explain more
An SCTP connection is called an association.

SCTP separates application data into chunks, each identified by a separate chunk header. 
These chunks are bundled into a single SCTP message that consists of an SCTP message header followed by several data chunks.
A key feature here is that the data chunks are independently identified with a separate header, thus a single SCTP message can contain data from separate streams of application data. For example one stream being text messages and another being the transfer of a file in a messaging application.
The advantage of this packet structure is that it means SCTP can support multi- streaming since it can send multiple data streams in parallel through a single SCTP association. 
%TODO: have a small section explaining head- of- line blocking and how SCTP avoids it 

Multi- streaming means that an application can transmit several independent streams of data in parallel. 

SCTP separates application data into chunks, each identified by a separate header. 
A single SCTP packet can contain several data chunks from different application data streams.

%TODO:cite some of the papers that have shown SCTP improvements compared to TCP 

\section{Congestion control}
\subsection{Loss- based congestion control}
\subsection{Delay- based congestion control}
\subsection{ECN}%TODO: Might not be necessary to explain?
\section{WebRTC Congestion controls}
Video data by nature is large in size so transmitting it creates a lot of traffic. 
This makes real- time communication challenging because it requires low latency in order to assure a good user experience. 

History and previous research [cite relevant stuff, like congestion collapse]has shown that protocols should employ mechanisms that limit the amount of data sent per second to a reasonable level in order to avoid congestion as well as keep the latency low.
\subsection{Google Congestion Control}
RTP by itself only provides simple end- to- end delivery services for multimedia[cite RTP standard], since real- time communication requires congestion control it must implemented on top of RTP. 
Chromium's WebRTC implementation uses an algorithm called Google Congestion Control \cite{draft-ietf-rmcat-gcc} to provide the mechanism.
It consists of two controllers, one loss- based and one delay- based. 
The loss- based controller located on the sender- side, uses loss rate, RTT and REMB[Cite REMB message definition] messages to compute a target sending bitrate. 
The delay- based controller can either be implemented on the receiver- side or sender- side.
It uses packet arrival info to compute a maximum bitrate which is passed to the loss- based controller. The actual sending rate is set to the minimum of the two bitrates.
% Might want to explain RTP terms like groups of packets and etc. 
\subsubsection{The loss- based controller}
The loss- based controller is run every time a feedback message from the receiver- side is received. 
If more than 10\% of packets have been lost when feedback is received the controller decreases the estimate. 
If less than 2\% is lost it will increase the estimate under the presumption that there is more bandwidth to utilize. 
Otherwise the estimate stays the same.
\subsubsection{The delay- based controller}
 The delay- based controller consists of several parts: pre- filtering, an arrival- time filter, an over- use detector and a rate controller. %% probably smart to have a figure of this

 Pre- filtering is used to make sure that channel outages, events unrelated to congestion are not interpreted as congestion.
Packets will naturally be delayed when a channel outage occurs so without this filter the algorithm would unnecessarily lower bitrate, thus lowering the quality of the communication for no reason.
A channel outage will cause the packets to be queued in network buffers, thus when the channel is restored the packets will arrive in bursts. 
The filter utilizes the fact that the packet groups will arrive in bursts during a channel outage and merges them under such conditions.

The arrival- time filter is responsible for calculating the queueing time variation which is an estimation of how the delay is developing at a certain time.
%TODO: fill this one with more info
The goal of the over- use detector is to produce a signal that drives the state of the remote rate controller. 
The goal of the over- use detector is to compare the queueing time variation obtained as output from the arrival- time filter with a threshold. If the estimate is above the threshold for a certain amount of time and not sinking it wil signal the rate contol.
%TODO: incorporate explaination of the remote rate controller as well as the related state machine and how it works
\subsubsection{Performance}
\subsection{NADA}
Network-Assisted Dynamic Adaptation (NADA) \cite{XiaoqingZhu2013NAUC} specified in \cite{rfc8698} is a proposed congestion control for WebRTC designed by Cisco.
The key design goal of NADA is to offer both fast adaption time to congestion whilst also being able to compete with TCP flows sharing the same bottleneck.
In order to achieve this, NADA combines loss, delay and ECN marked packets into a composite network congestion signal.
\subsubsection{System overview}
%TODO: probably nice with a figure of the system
We will first give a basic overview of the central components of the system, then explain each of them afterwards more thoroughly.
%Live video encoder
The live video encoder is a component responsible for encoding incoming raw video frames into RTP packets.
It tries to output RTP packets at a rate as close as possible to the target input rate $R_v$ decided by the rate control. 
The actual output rate is denoted by $R_o$ which will be a number within a range $[R_{min},R_{max}]$ depending on the video scene complexity and can change over time.
The value of the output rate $R_o$ may fluctuate randomly around the input target rate $R_v$.
On top of that the live video encoder is only capable of reacting to changes in $R_v$ over long time intervals that might be in the order of seconds.
The encoder's typical reaction time is denoted by $T_v$.
%NADA sending agent
The NADA sending agent has the responsibility of calculating a reference rate $R_n$ based on the composite congestion signal reported by the receiver. 
The reference rate calculated by the sending agent is then used to regulate the video sending rate $R_s$.
However there will be a difference between actual video encoder ouput and regulated send rate, a rate shaping buffer is used to absorb that difference.
The size of the buffer $L_s$ along with $R_n$ determine the video encoder target rate and the sending rate. 
%Network node
NADA is designed to work with nodes in different operation modes and it support many different queueing modes.
%NADA receiving agent
The NADA receiving agent is responsible for calculating the composite congestion signal used by the sending agent to regulate the sending rate. 
To do this it uses derived one- way delay of each packet, loss events and ECN markings.
The one- way delay $D_n$ is derived from RTP header timestamps while ECN markings are extracted from the IP headers. 
The resulting composite signal is in the form of an "equivalent delay" $~d_n$.
A time smoothed version $x_n$ of the signal is periodically reported back to the sending agent through RTCP messages.

%TODO: a table of symbols might be nice
\subsubsection{Receiver agent}
The receiver agent has four main tasks: 
a) Monitor one- way delay, packet loss and gather ECN marking statistics. 
b) Aggregate the different congestion signals into a composite network congestion signal. 
c) Calculate a time- smoothed value of the composite congestion signal. 
d) Send periodic reports of congestion to the sender.
\paragraph{a. Monitoring one- way delay, packet loss and gathering ECM marking statistics}
\paragraph{b. Aggregating different congestion signals into a composite network congestion signal}
\paragraph{c. Calculating a time- smoothed value}
\paragraph{d. Sending reports of congestion to the sender}

\subsubsection{Sender agent}
\subsubsection{Performance}
\subsubsection{Usage}

\subsection{SCReAM}

\section{Coupled Congestion Control}
\subsection{Problems with combined controls}
\paragraph{}
There are inherent differences in how quickly different types of congestion control react to congestion and combining them can therefore easily lead to unintentional side- effects. 
This is a known issue especially when it comes to combining loss- based controls with delay- based controls.
It has been shown to lead to a competition between the flows resulting in spikes in queueing delay and packet loss. 
%TODO: Any sources to cite for this, e.g. in regards to WebRTC?
The main issue stems from the fact that delay always happens earlier than loss.
The reason is that the first thing that happens when there is congestion is that the bottleneck queues will start filling, thus making packets more delayed but not necessarily dropped until the queue is full or close to full.
%TODO: some article showing problems with combining loss-based and delay- based
Intuitively, this means that delay is observable earlier than loss when there is congestion.
The consequence of all this is that the delay- based control will decrease it's sending rate sooner than the loss- based will. 
Such behaviour leads to a very bad dynamic where the delay- based control lowers the bitrate at such an early stage of congestion that the loss- based never experiences packet loss and thus keeps increasing its send rate. 
The final result is that the delay- based control will get a smaller and smaller share of the available bandwidth because the loss- based control keeps increasing while the delay- based keeps backing down because of the congestion caused by the still- increasing traffic from the loss- based controller.
\paragraph{Coupling the controls}
A possible solution is to have the controllers cooperate by sharing their information, given that they are both located on the same host and are travelling the same path across the network i.e. coupling the flows.
Firstly, such a mechanism could benefit both controllers by giving them more information, and different types of information to base their decisions on. 
Secondly this could be used to ensure that the loss- based controllers behaves more fairly to controllers.
One could for instance make sure that the loss- based control also decreases when the delay- based controller decreases the send rate.
One mechanism for coupling is "The Congestion Manager" (CM) \cite{rfc3124}.
CM couples flows by offering a single shared congestion controller for all the flows instead.
The downside is that it is considered quite hard to implement because it requires an extra congestion controller and strips away all per- connection congestion control functionality, which is a drastic change.
%TODO: Should actually read RFC3124 yourself
%TODO: Maybe discuss Ensemble TCP?
A newer solution called "Coupled Congestion Control" \cite{rfc8699} combines congestion controls travelling over the same bottleneck while at the same time being easier to implement than the congestion manager.
As opposed to The Congestion Manager, Coupled Congestion Control tries to utilize the congestion control algorithms of the coupled flows by sharing the information they gather among them instead of completely removing them. 
The mechanism has aready shown promise in \cite{10.1145/2740070.2630089, 7502803} when implemented with homogenous congestion controls but has not been tested on heterogenous congestion controls. 
%TODO: read about how it performs and cite/mention it around here
%TODO: Explain why this could prove useful for WebRTC?
\paragraph{Coupled Congestion Control Architecture}
The design philosophy of Coupled Congestion Control is that the amount of required changes to existing applications should be minimal. 
The system consists of three elements, Shared Bottleneck Detection(SBD), Flow State Exchange(FSE) and the flows. 

\paragraph{Managing flows}

When a flow starts it will register itself with the FSE and SBD, when it stops it will deregister from the FSE and carry out an UPDATE function call every time their congestion controller calculates a new rate.
When a flow registers itself the SBD will assign it to a Flow Group by giving it a Flow Group Identifier.
A flow group is defined as a group of flows that share the same bottleneck and thus should exchange information with each other. 
The SBD is responsible for reassigning a flow to a different FG whenever the bottleneck changes.

\subsection{The Flow State Exchange}
The FSE can be described as a manager that maintains information exchanged between the flows and calculates a bit rate for each flow based on all the information gathered. 

It can be implemented in two ways: \textit{active} or \textit{passive}.
In the active version, the FSE will actively initiate communication with each flow and SBD. 
The passive version does not actively initiate communication and only has the task of internal state maintenance.
Generally the FSE keeps a list of all flows that have registered with it and for each flow the FSE will store the following:
\begin{itemize}
    \item A unique number f to identify the flow.
    \item The Flow Group Identifier (FGI).
    \item The priority value P(f).
    \item The rate used by the flow which is calculted by the FSE in bits per second FSE\_R(f).
    \item The desired rate of the flow, DR(f).
\end{itemize}

The priority value P is used to calculate the flow's priority portion out of the sum of all priority values.
The desired rate might be smaller than the calculated rate, e.g. because the application wants to limit the flow or simply does not have enough data to send. 
If there is no desired rate value given by the flow it should just be set to the sending rate provided by the flows congestion control.

For each FG the FSE keeps a few static variables:
\begin{itemize}
    \item The sum S\_CR of calculated rates for all flows in th FG.
    \item The sum S\_P of all priorities in the FG.
    \item The total leftover rate TLO. This is the sum of leftover rate by rates limited by desired rate.
    \item Aggregate rate AR given to flows that are not limited by desired rate. 
\end{itemize}

Every time a flow's congestion control normally would update the flow's rate they carry out an UPDATE call to FSE instead. 
Through the UPDATE call they provide their newly calculated rate and optionally a desired rate. 
Then FSE calculates rates for all the flows and sends them back. 
When a flow f starts, FSE\_R is initialized with the initial rate calculated by f's congestion controller. 
After the SBD assigns the flow to an FG, it adds its FSE\_R to S\_CR.
The desired rate is smaller than the calculated rate when the flow is limited by an application, otherwise it will be the same as the calculated rate.

\subsubsection{Active FSE}
In the active version FSE recalculates rates and notifies all the other flows in the FG as well whenever there is an UPDATE call from a single flow. 

% What happens when a flow start
% What happens when a flow UPDATES
\begin{figure}
\begin{center}
    \begin{tabular}{|c|c|}
        \hline
        CC_R & The rate recevied from a flow's congestion controller when it calls UPDATE \\
        new\_DR & The desired rate received from a flow when it calls UPDATE \\
        FSE\_R & The rate allocated to a flow from the FSE \\
        S\_CR & The total sum of calculated rates for all flows in the samme FG \\ 
        FG & A group of flows sharing the same bottleneck \\
        P & The priority of a flow, received from the flow's congestion controller \\
        S\_P & The sum of all prioritites in a flow group \\
        DELTA & A variable used to calculate the difference between CC\_R and the previous FSE\_R \\  
        \hline
    \end{tabular}
\end{center}
\caption{Variables used in active FSE}
\label{active_fse_variables}
\end{figure}

In \cite{rfc8699} there are two examples of active FSE algorithms outlined.
In figure \ref{active_fse_variables} the variables used in both algorithms are outlined. 

\paragraph{Example algorithm 1}
The first algorithm was designed to be the simplest possible method for assigning rates according to prioritites of flows. 
It consists of three steps:
1. When a flow f starts, it registers itselfi with SBD and FSE. FSE\_R(f) is initialized with f's congestion controllers initial rate. 
SBD will also give f a correct FGI.
After having received its FGI the FSE adds the FSE\_R(f) to the flow group's corresponding S\_CR.
2. When a flow f stops or pauses it gets removed from the list of registered flows.
3. Every time a flow f's congestion controller updates the send rate CC\_R(f), i make an UPDATE call to the FSE.
The UPDATE function completes four basic tasks (a- d) in order to calculate the new send rate for all flows in the same FG.
A flow's UPDATE function uses three local (per- flow) temporary variables: S\_P, TLO and AR.
UPDATE's tasks:
a) First it has to update S\_CR, this is done by adding the difference between CC\_R(f) and the previous FSE\_R(F) to the S\_CR.
b) Then it calculates a new S\_P value and initializes FSE\_R(f) values of all flows in the FG to 0.
c) In the third step it distributes the S\_CR among all the flows in the FG by calculating new FSE\_R values, while also ensuring that each flow's desired rate is not surpassed.
d) Lastly the FSE actively distributes the newly calculated FSE\_R values to all the flows in the FG.

%TODO: finish pseudocode
\begin{algorithm}
    \caption{Active FSE - Example 1}
    \begin{algorithmic}
        \COMMENT{3a}
        \STATE $S\_CR \leftarrow S\_CR + CC\_R(f) - FSE\_R(f)$ 
        \COMMENT{3b}
        %TODO: finish this 
    \end{algorithmic}
\end{algorithm}


Even though algorithm 1 is simple and intuitive, it is shown to give both higher packet loss and queueing delay in \cite{10.1145/2740070.2630089} when tested with two heterogenous controls coupled. 
The authors concluded that the reason for these unsatisfactory results happened because the FSE \textit{de-synchronizes} the flows.
To illustrate the problem, consider some arbitrary congestion control that halves its send rate when experiencing congestion.
The problem essentially was that normally without FSE, two flows will often naturally get synchronized and halve their rate at the same time, thus reducing the total rate by half.
Now, if we consider the two flows coupled with the FSE again, when one tries to increase its rate and experiences congestion it halves its rate which only reduces the total rate a quarter.
As a consequence there was a new active FSE algorithm made in order to fix the loss ratio and average queue growth.
\paragraph{Example algorithm 2}
The second algorithm aims to emulate a behavior similar what happens when flows get synchronized.
It does this by proportionally reducing the aggregate rate on congestion.
%TODO: Summarize the algorithm
In the second algorithm step 3a is altered by introducing the local variable DELTA which is used to calculate the difference between CC\_R and previously stored FSE\_R. 
To prevent flows from either ignoring congestion or overreacting, a timer is used to stop any flow form changing their directly after a shared rate reduction caused by a congestion event. 
The timer starts at two RTT's of the flow that experienced the congestion event.
The reasoning for using two RTT's is that it is assumed that a congestion event may last up to one RTT for the flow, with the extra RTT added in order to compensate for fluctuations in measured RTT value.
Except for step 3a where S\_CR is updated based on DELTA, the rest of the algorithm remains the same as algorithm 1.
%TODO: write pesudocode for step 3a

\paragraph{Passive FSE}
In the passive version of FSE the rate is not calculated for other flow's than the one making the UPDATE call, as opposed to active FSE.
This might be considered easier to implement but when flows have different RTT's, the ones with shorter RTT will update more often which could produce unwanted side- effects.
The problem is even more significant in situations where a congestion controller's convergence is dependent on RTT. %TODO: explain why maybe?
In the passive version of FSE the TLO variable is static per FG and is initialized to 0. 
1. When a flow f starts it registers itself til SBD and FSE. FSE\_R(f) and DR(f) is initialized with the congestions controller's initial rate. 
Additionally, S\_CR is limited to increase or decrease as conservatively as the flows's congestion controller decides in order to avoid sudden rate jumps.
2. When a flow f stops or pauses, DR(f) is set to 0 and P(f) is set to -1. %TODO:Why -1?
3. Every time a flow f's congestion controller calculates a new send rate CC\_R(f) it calls UPDATE.
The UPDATE function uses some local(per- flow) temporary variables that are initialized to 0: DELTA, new\_S\_CR and S\_P.
The UPDATE function of a flow f completes tasks a- e:
a) For all flows in the same FG it calculates the sum of all calculated rates new\_S\_CR.
Afterwards it calculates DELTA as the difference between FSE\_R(f) and CC\_R(f).
b) It updates S\_CR, FSE\_R(f) and DR(f).
c) It calculates TLO if there is any leftover rate, removes terminated flows from the list of flows and calculates the sum of all priorities, S\_P.
d) It calculates the send rate, Rate(f) which is the minimum of DR(f) and prioritized share of S\_CR plus any leftovers from other flows.
e) Lastly it updates DR(f) and FSE\_R with Rate(f).
%TODO: explain more?

\section{Shared Bottleneck Detection}
The SBD is an entity that is responsible for determining which flows are traversing the same bottleneck. 
In \cite{rfc8699} three methods for deriving if flows share the same bottleneck are mentioned.

\subsection{Multiplexed flows}
One way is through comparing multiplexed flows. 
%TODO: might want to explain the term five- tuple
Since the flows with the same five- tuple will be routed along the same path, SBD can assume that they share the same bottleneck. 
However this method cannot be used for coupled congestion controllers with one sender talking to multiple receivers, given that they will not have the same five- tuple. 
Since WebRTC uses both SRTP and SCTP multiplexed on UDP, this ensures that they have the same five- tuple and that the first method will work. 
\subsection{Measurement}
One might also use measurements of e.g. delay and loss and look at correlations to derive if flows have a shared bottleneck.
\subsection{Configuration}


\chapter{Design}

\chapter{Implementation}                 


\chapter{Evaluation}
\section{Testbed}
\section{Experiments}

\chapter{Conclusion}                    
\section{Research Findings}
\section{Further work}
\section{Closing remarks}

\backmatter{}
\printbibliography
\end{document}
